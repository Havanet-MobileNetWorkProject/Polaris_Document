\documentclass{report}
\usepackage{ptext}
\usepackage{lipsum}
\usepackage{graphicx}
\usepackage{ptext}

\input{Boostan-UserManual}

\newword{Abstraction}{Abstraction}
{انتزاع}{}

\newword{Abstract}{Abstract}
{انتزاعی}{}

\newword{AbsoluteMinimum}{Absolute Minimum}
{کمینه مطلق}{}


\newword{AcceptableCell}{Acceptable Cell}
{سلول پذیرفتنی}{سلول‌های پذیرفتنی}

\newword{AccessBurst}{Access Burst}
{توده دسترسی}{توده‌های دسترسی}


%%% S
\newword{Sample}{Sample}
{نمونه}{نمونه‌ها}

\newword{SamplePath}{Sample Path}
{نمونه مسیر}{}

\newword{SampleSpace}{Sample Space}
{فضای نمونه}{فضای نمونه‌ها}
\newacronym{ACK}{ACK}{Acknowledgement}

\newacronym{ACI}{ACI}{Application Control Interface}

\newacronym{ACIR}{ACIR}{Adjacent Channel Interference Ratio}

\newacronym{ACLC}{ACLC}{Adaptive Configuration of Logical Channels}

\newacronym{ACLP}{ACLP}{Adjacent Channel Leakage Power}

\title{مستند راه اندازی پروژه نهایی درس شبکه های تلفن همراه}
\type{مستند }
\author{زهرا دهقان\\اسماء حمید\\فاطمه شرح دهی مقدم}

\logofile{Pic/logo}


\begin{document}

\pagenumbering{gobble}
\Godpage
\maketitle
\pagenumbering{arabic}
\tableofcontents


\chapter{مقدمه}








\chapter{ \lr{Server-side Component}}

\chapter{ \lr{Web Application}}

\chapter{ \lr{Android Mobile}}

\section{تنظیم IP بک‌اند در اپلیکیشن اندروید}


برای اتصال اپلیکیشن به بک‌اند در محیط local، بک‌اند و موبایل باید در یک شبکه باشند. سپس آدرس IP سیستم میزبان بک‌اند را با دستور \textbf{ipconfig} در CMD پیدا کنید:

\vspace{0.3cm}

آدرس \lr{IPV4} را یادداشت کرده و در دو محل زیر جایگزین کنید:

\begin{enumerate}
	\item \textbf{SharedViewModel.kt}\\
	مسیر: \\
		\begin{latin}
	\texttt{app/src/main/java/com/example/Havanet/viewmodels/SharedViewModel.kt}\\
\end{latin}
مقدار موجود در خط شماره پانزده، عبارت از:
\begin{latin}
\begin{lstlisting}[mathescape=true, numbers=left, firstnumber=15]
val ip = "10.13.148.180"
\end{lstlisting}
\end{latin}
	را با IP جدید جایگزین کنید.
	
	\item \textbf{network\_security\_config.xml}\\
	مسیر: \\
		\begin{latin}
	\texttt{app/src/main/res/xml/network\_security\_config.xml}\\
		\end{latin}
	مقدار موجود در خط شماره چهار، عبارت از:
\begin{latin}
	\begin{lstlisting}[mathescape=true, numbers=left, firstnumber=4]
<domain includeSubdomains="true">10.13.148.180</domain>
	\end{lstlisting}
\end{latin}
	را با IP جدید جایگزین کنید.
\end{enumerate}



 
\end{document}



