\documentclass{report}
\usepackage{ptext}
\usepackage{lipsum}
\usepackage{graphicx}
\usepackage{ptext}

\input{Boostan-UserManual}

\input{Chapters/BWords}
\input{Chapters/abbre}

\title{گزارش پروژه نهایی درس شبکه های تلفن همراه}
\type{گزارش }
\author{زهرا دهقان\\اسماء حمید\\فاطمه شرح دهی مقدم}

\logofile{Pic/logo}


\begin{document}

\pagenumbering{gobble}
\Godpage
\maketitle
\pagenumbering{arabic}
\tableofcontents


\chapter{مقدمه}








\chapter{ \lr{Polaris Server}}

\chapter{Polaris Server}
\section{Authentication}

\subsection{Register API (ثبت‌نام کاربر)}
\textbf{مدل (User Model)} \\
مدل کاربر برای سامانه Polaris به شکل زیر طراحی شده است:

\begin{itemize}
  \item \lr{phone}: شماره تلفن کاربر (یونیک و به‌عنوان نام کاربری استفاده می‌شود).
  \item \lr{role}: نقش کاربر که می‌تواند \lr{user} یا \lr{admin} باشد (پیش‌فرض: \lr{user}).
  \item \lr{is\_active}: تعیین فعال بودن کاربر.
  \item \lr{is\_staff}: برای تشخیص دسترسی مدیریتی.
\end{itemize}

کاربران عادی نقش \lr{user} دارند و مدیران (\lr{superuser}) نقش \lr{admin}. \\
همچنین یک \lr{UserManager} پیاده‌سازی شده تا:
\begin{itemize}
  \item تابع \lr{create\_user}: کاربر عادی بسازد.
  \item تابع \lr{create\_superuser}: ادمین بسازد.
\end{itemize}

\textbf{Serializer (RegisterSerializer)} \\
کلاس \lr{RegisterSerializer} برای اعتبارسنجی داده‌های ورودی ثبت‌نام استفاده می‌شود:
\begin{itemize}
  \item فقط دو فیلد دریافت می‌کند: \lr{phone} و \lr{password}.
  \item \lr{password} فقط نوشتنی (\lr{write\_only}) است.
  \item تابع \lr{create}: از متد \lr{create\_user} در \lr{UserManager} استفاده می‌کند.
\end{itemize}

\textbf{View (RegisterView)} \\
کلاس \lr{RegisterView} یک \lr{APIView} است که درخواست‌های POST را پردازش می‌کند:
\begin{itemize}
  \item داده‌های ورودی توسط \lr{RegisterSerializer} بررسی می‌شود.
  \item اگر معتبر بود: کاربر ساخته می‌شود و توکن‌های JWT صادر می‌گردد.
  \item خروجی شامل توکن‌ها + اطلاعات کاربر است.
\end{itemize}

\textbf{ورودی (POST /auth/signup/)}:
\begin{lstlisting}[language=json]
{
  "phone": "09123456789",
  "password": "test1234"
}
\end{lstlisting}

\textbf{خروجی موفق (201)}:
\begin{lstlisting}[language=json]
{
  "refresh": "eyJ0eXAiOiJKV1QiLCJhbGciOiJIUzI1...",
  "access": "eyJ0eXAiOiJKV1QiLCJhbGciOiJIUzI1...",
  "user": {
    "id": 1,
    "phone": "09123456789",
    "role": "user"
  }
}
\end{lstlisting}

\textbf{خروجی ناموفق (400)}:
\begin{lstlisting}[language=json]
{
  "phone": ["This field must be unique."]
}
\end{lstlisting}
\subsubsection{سیگنال پس از ثبت‌نام کاربر}
پس از ایجاد کاربر جدید (تنها در صورتی که نقش کاربر \lr{user} باشد)، یک سیگنال \lr{post\_save} از طرف جنگو فراخوانی می‌شود. این سیگنال وظیفه دارد به صورت خودکار مقادیر پیش‌فرض \lr{ThresholdParameter} و \lr{ThresholdLevel} را برای کاربر ایجاد کند. 

این مقادیر شامل پارامترهای مختلف کیفی و کمی برای تکنولوژی‌های \lr{2G}, \lr{3G}, \lr{4G} و \lr{5G} هستند. برای هر پارامتر، چند سطح (Level) با بازه‌های عددی و رنگ مشخص تعریف می‌شود.  

کلیت مفهوم \lr{Threshold} در بخش‌های بعدی توضیح داده خواهد شد، اما در اینجا تنها نحوه ایجاد خودکار آن پس از ثبت‌نام کاربر شرح داده می‌شود. 

\textbf{کد سیگنال (create\_default\_thresholds)}:
\begin{lstlisting}[language=python]
from django.db.models.signals import post_save
from django.dispatch import receiver
from authentication.models import User
from threshold.models import ThresholdParameter, ThresholdLevel
from threshold.defaults import default_thresholds 

@receiver(post_save, sender=User)
def create_default_thresholds(sender, instance, created, **kwargs):
    if created and instance.role == "user": 
        for tech, params in default_thresholds.items():
            for p in params:
                param = ThresholdParameter.objects.create(
                    user=instance,
                    name=p["name"],
                    technology=tech,
                    signal_type=p["signal_type"],
                )
                for lvl in p["levels"]:
                    ThresholdLevel.objects.create(
                        parameter=param,
                        level=lvl["level"],
                        color=lvl["color"],
                        min_value=lvl["min"],
                        max_value=lvl["max"],
                    )
\end{lstlisting}

\textbf{نمونه‌ای از مقادیر پیش‌فرض (\lr{default\_thresholds})}:
\begin{lstlisting}[language=python]
default_thresholds = {
    "2G": [
        {
            "name": "rxlev",
            "signal_type": "quantity",
            "levels": [
                {"level": 1, "color": "#FF0000", "min": -110, "max": -100},
                {"level": 2, "color": "#FFFF00", "min": -100, "max": -80},
                {"level": 3, "color": "#008000", "min": -80, "max": -50},
            ],
        }
    ],
    "3G": [
        {
            "name": "rscp",
            "signal_type": "quantity",
            "levels": [
                {"level": 1, "color": "#FF0000", "min": -120, "max": -110},
                {"level": 2, "color": "#FFFF00", "min": -110, "max": -90},
                {"level": 3, "color": "#008000", "min": -90, "max": -60},
            ],
        }
    ],
    "4G": [
        {
            "name": "rsrp",
            "signal_type": "quantity",
            "levels": [
                {"level": 1, "color": "#FF0000", "min": -140, "max": -120},
                {"level": 2, "color": "#FFA500", "min": -120, "max": -110},
                {"level": 3, "color": "#008000", "min": -110, "max": -80},
            ],
        },
        {
            "name": "rsrq",
            "signal_type": "quality",
            "levels": [
                {"level": 1, "color": "#FF0000", "min": -20, "max": -15},
                {"level": 2, "color": "#FFFF00", "min": -15, "max": -10},
                {"level": 3, "color": "#008000", "min": -10, "max": -3},
            ],
        },
    ],
    "5G": [
        {
            "name": "rsrp",
            "signal_type": "quantity",
            "levels": [
                {"level": 1, "color": "#FF0000", "min": -140, "max": -120},
                {"level": 2, "color": "#FFA500", "min": -120, "max": -110},
                {"level": 3, "color": "#008000", "min": -110, "max": -80},
            ],
        },
        {
            "name": "rsrq",
            "signal_type": "quality",
            "levels": [
                {"level": 1, "color": "#FF0000", "min": -20, "max": -15},
                {"level": 2, "color": "#FFFF00", "min": -15, "max": -10},
                {"level": 3, "color": "#008000", "min": -10, "max": -3},
            ],
        },
    ],
}
\end{lstlisting}


\subsection{Login API (ورود کاربر)}
\textbf{View (LoginView)} \\
\begin{itemize}
  \item درخواست POST می‌گیرد.
  \item شماره تلفن و رمز عبور را بررسی می‌کند.
  \item در صورت معتبر بودن، توکن‌های JWT صادر می‌کند.
  \item اطلاعات کاربر بازگردانده می‌شود.
\end{itemize}

\textbf{ورودی (POST /auth/login/)}:
\begin{lstlisting}[language=json]
{
  "phone": "09123456789",
  "password": "test1234"
}
\end{lstlisting}

\textbf{خروجی موفق (200)}:
\begin{lstlisting}[language=json]
{
  "refresh": "eyJ0eXAiOiJKV1QiLCJh...",
  "access": "eyJ0eXAiOiJKV1QiLCJh...",
  "user": {
    "id": 1,
    "phone": "09123456789",
    "role": "user"
  }
}
\end{lstlisting}

\textbf{خروجی ناموفق (401)}:
\begin{lstlisting}[language=json]
{
  "error": "Invalid credentials"
}
\end{lstlisting}


\subsection{Logout API (خروج کاربر)}
\textbf{View (LogoutView)} \\
\begin{itemize}
  \item درخواست POST دریافت می‌کند.
  \item کاربر باید توکن \lr{refresh} را ارسال کند.
  \item توکن بلاک‌لیست می‌شود.
\end{itemize}

\textbf{ورودی (POST /auth/logout/)}:
\begin{lstlisting}[language=json]
{
  "refresh": "eyJ0eXAiOiJKV1QiLCJh..."
}
\end{lstlisting}

\textbf{خروجی موفق (205)}: (بدون بدنه) \\
\textbf{خروجی ناموفق (400)}: (بدون بدنه)


\subsection{Profile API (نمایه کاربر)}
\textbf{Serializer (UserSerializer)} \\
برای نمایش اطلاعات کاربر از \lr{UserSerializer} استفاده می‌شود (شامل \lr{id}, \lr{phone}, \lr{role}).

\textbf{View (ProfileView)} \\
\begin{itemize}
  \item درخواست GET دریافت می‌کند.
  \item نیاز به احراز هویت دارد.
  \item اطلاعات کاربر لاگین‌شده بازگردانده می‌شود.
\end{itemize}

\textbf{ورودی (GET /auth/profile/)}: بدون بدنه (توکن در Header). \\

\textbf{خروجی موفق (200)}:
\begin{lstlisting}[language=json]
{
  "id": 1,
  "phone": "09123456789",
  "role": "user"
}
\end{lstlisting}

\textbf{خروجی ناموفق (401)}:
\begin{lstlisting}[language=json]
{
  "detail": "Authentication credentials were not provided."
}
\end{lstlisting}







\chapter{ \lr{Polaris Client}}

\section{ \lr{Web Application}}

\section{ \lr{Android Mobile Client}}

\chapter{تحلیل و نتجه گیری}
 
\end{document}



